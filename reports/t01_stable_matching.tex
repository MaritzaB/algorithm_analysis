\documentclass{article}
\usepackage[spanish,es-tabla,es-nodecimaldot]{babel}
\usepackage{algpseudocode}
\usepackage{algorithm}
\setlength{\parindent}{0pt}
\setlength{\parskip}{1em}
\floatname{algorithm}{Algoritmo}

\begin{document}
\title{Análisis y diseño de algoritmos. \\ Tarea 01}
\author{Ana Maritza Bello Yañez}
\maketitle

\section*{Problema 1.3}
Supongamos los siguientes raitings de programación para la televisora $A$:

\begin{equation}
    S = \{7.0, 3.9\}
\end{equation}

Y para la televisora $B$:

\begin{equation}
    T = \{5.8, 2.5\}
\end{equation}

Una instancia que generaría \textit{stable matching} sería el siguiente, donde
la televisora $A$, gana dos slots:
 
\begin{equation}
    (S,T) = \{(7.0,5.8),(3.9,2.5)\}
\end{equation}

Ya que la televisora A gana un slot cuando 7 compite con 5.8, y el otro cuando
3.9 compite con 2.5.

Pero la televisora $B$ podría cambiar su horario a $T'$:

\begin{equation}
    T' = \{2.5, 5.8\}
\end{equation}

Con el fin de que 5.8 le gane a 3.9.
Por lo que en $(S,T')$ existe inestabilidad.

\section*{Problema 1.4}

\begin{algorithm}
\caption{Algoritmo para asociar estudiantes a hospitales}\label{alg:cap}
\begin{algorithmic}
\Require Inicialmente todos los $h \in H$ and $s \in S$ están libres
\While{Haya un hospital $h$ que tenga lugares y no le haya propuesto a cada estudiante}
    \State Escoge al hospital $h$
    \State Sea $s$ el estudiante mejor clasificado en la lista de preferencias de $h'$ a quien $h$ aún no le ha propuesto residencia
    \If{ $s$ está libre}
        \State $(h,s)$ hacen pareja
    \Else{$s$ es actualmente residente de $h'$}
        \If{$s$ prefiere a $h'$ sobre $h$}
            \State el lugar en $h$ permanece libre
        \Else{$s$ prefiere a $h$ sobre $h'$}
            \State $(h,s)$ hacen pareja
            \State el lugar en $h'$ se libera
        \EndIf
    \EndIf
\EndWhile
\State Regresa el conjunto de parejas de hospitales y residentes $C$
\end{algorithmic}
\end{algorithm}

\subsection*{Complejidad del problema}
La complejidad del problema es $\phi(n^2)$.

El algoritmo termina en a lo máximo $m*(m-n)$ iteraciones. Siendo $m$ en número de
lugares ofertados por los hospitales y $n$ el número de residentes.

Por cada loop del ciclo while, un hospital oferta un lugar a un estudiante. Pero
solo hay $m$ cantidad de lugares para ofertar, por lo que solo hay $n²$ posibles
propuestas.

\subsection*{Demostración por contradicción}

Demuestra que siempre hay una asignación estable de estudiantes a hospitales.

Supongamos que en el conjunto $C = {(h_1,s_1),(h_2,s_2)}$ existe una pareja
inestable $(h_1,s_2)$:
1. Esto quiere decir que $h_1$ nunca le propuso a $(s_2)$, por lo que prefiere a
su pareja actual por lo que sería una pareja estable.

ó 

2. $h_1$ le propuso a $(s_2)$, por lo que si no están juntos entonces el
estudiante $s_2$ rechazó a $h_1$, por lo que el estudiante prefiere su
residencia actual sobre $h_1$, y por lo tanto es una pareja estable.


\section*{Problema 1.5a}

\begin{algorithm}
\caption{Algoritmo para asociar a hombres y mujeres con una lista de preferencia
en la que pueden haber empates}\label{alg:cap}
    \begin{algorithmic}
    \Require Inicialmente todos los $m \in M$ y $w \in W$ están libres
    \While{Haya un hombre $m$ que esté libre y que no se le haya propuesto a cada mujer}
        \State Escoge al hombre $m$
        \State Sea $w$ la mujer mejor clasificada en la lista de preferencias de $m'$ a
        quien $m$ aún no se le ha propuesto
        \If{ $w$ está libre}
            \State $(m,w)$ se vuelven pareja
        \Else{$w$ está comprometida con $m'$}
            \If{$w$ prefiere a $m'$ sobre $m$}
                \State $m$ permanece libre
            \Else{$w$ prefiere a $m$ sobre $m'$}
                \State $(m,w)$ se vuelven pareja
                \State $m'$ se queda libre
            \EndIf
        \EndIf
    \EndWhile
    \State Regresa el conjunto de parejas comprometidas $S$
    \end{algorithmic}
\end{algorithm}

\subsection*{Complejidad del problema}
La complejidad del problema es $\phi(n^2)$.
    
El algoritmo termina en a lo máximo $n^2$ iteraciones.
    
Por cada loop del ciclo while, un hombre le hace un propuesta a una mujer. Por
lo que solo hay $n^2$ iterciones posibles. 
    
\subsection*{Demostración por contradicción}

Inestabilidad fuerte: Consiste en un hombre $m$ y una mujer $w$ tal que $m$ y
$w$ se prefieren sobre su pareja actual.
    
¿Existe un \textit{perfect matching} sin inestabilidad fuerte?
    
Supongamos que en el conjunto $S = {(m_1,w_1),(m_2,w_2)}$ existe una pareja
inestable $(m_1,w_2)$:
1. Esto quiere decir que $m_1$ nunca le propuso a $(w_2)$, por lo que prefiere a
su pareja actual por lo que sería una pareja estable.

ó 

2. $m_1$ le propuso a $w_2$, por lo que si no están juntos entonces $w_2$
rechazó a $m_1$, por lo prefiere a su pareja actual sobre $m_1$, y por lo
tanto es una pareja estable.

\section*{Problema 1.7}

\begin{algorithm}
\caption{Algoritmo para encontrar cambios válidos en el flujo de salidas y
entradas de cajas de conexión}\label{alg:cap}
    \begin{algorithmic}
        \Require Inicialmente todos las entradas $e \in E$ y salidas $s \in S$ están libres
        \Require Invertimos el orden de prioridad de las salidas
        \While{Haya una entrada $e$ sin conectar y no haya propuesto a cada salida}
        \State Escoge una entrada $e$
        \State Sea $s$ la primera salida en la lista de preferencias de $e'$ a
        la que $e$ aún no se le ha propuesto
        \If{ $s$ está libre}
            \State $(e,s)$ se pueden conectar
        \Else{$s$ ya está conectada con $e'$}
            \If{$s$ prefiere a $e'$ sobre $e$}
                \State $e$ permanece libre
            \Else{$s$ prefiere a $e$ sobre $e'$}
                \State $(e,s)$ se pueden conectar
                \State $s'$ se queda libre
            \EndIf
        \EndIf
    \EndWhile
    \State Regresa el conjunto de parejas que pueden ser conexión $C$
    \end{algorithmic}
\end{algorithm}

\subsection*{Complejidad del problema}
La complejidad del problema es $\phi(n^2)$.
    
El algoritmo termina en a lo máximo $n^2$ iteraciones.
    
Por cada loop del ciclo while, un entrada le hace un propuesta a una salida. Por
lo que solo hay $n^2$ iterciones posibles. 
    
\subsection*{Demostración por contradicción}

Inestabilidad fuerte: Consiste en una entrada $e$ y una salida $s$ tal que $e$ y
$s$ se prefieren sobre su pareja actual.
    
¿Existe un \textit{perfect matching} sin inestabilidad fuerte?
    
Supongamos que en el conjunto $C = {(e_1,s_1),(e_2,s_2)}$ existe una pareja
inestable $(m_1,w_2)$:
1. Esto quiere decir que $e_1$ nunca le propuso a $(s_2)$, por lo que prefiere a
su pareja actual por lo que sería una pareja estable.

ó 

2. $e_1$ le propuso a $s_2$, por lo que si no están juntos entonces $s_2$
rechazó a $s_1$, por lo prefiere a su pareja actual sobre $e_1$, y por lo
tanto es una pareja estable.


\end{document}
