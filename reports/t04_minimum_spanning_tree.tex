\documentclass{article}
\usepackage[spanish,es-tabla,es-nodecimaldot]{babel}
\usepackage{amsmath}
\usepackage[shortlabels]{enumitem}

\setlength{\parindent}{0pt}
\setlength{\parskip}{1em}


\begin{document}
\title{Análisis y diseño de algoritmos. \\ Tarea 04}
\author{Ana Maritza Bello Yañez}
\maketitle

\section*{Problema 4.3}
% Prueba por inducción
% Usted está asesorando a una empresa de camiones que hace una gran cantidad de
% envíos comerciales de paquetes entre Nueva York y Boston. El volumen es tan alto
% que tienen que enviar una cantidad de camiones cada día entre las dos
% ubicaciones. Los camiones tienen un límite fijo W sobre la cantidad máxima de
% peso que pueden transportar. Las cajas llegan a la estación de Nueva York una a
% una, y cada paquete i tiene un peso w_i. La estación de camiones es bastante
% pequeña, por lo que como máximo puede haber un camión en la estación en
% cualquier momento. La política de la empresa requiere que las cajas se envíen en
% el orden en que llegan; de lo contrario, un cliente podría molestarse al ver que
% una caja que llegó después de la suya llegó a Boston más rápido. Por el momento,
% la compañía está utilizando un algoritmo codicioso simple para empacar: empacan
% las cajas en el orden en que llegan, y cuando la siguiente caja no cabe, envían
% el camión por su camino.
% 
% Pero se preguntan si podrían estar usando demasiados camiones y quieren su
% opinión sobre si la situación se puede mejorar. Así es como están pensando. Tal
% vez se podría disminuir la cantidad de camiones necesarios enviando a veces un
% camión que estaba menos lleno, y de esta manera permitir que los próximos
% camiones estén mejor embalados.
% 
% Demuestre que, para un conjunto dado de cajas con pesos específicos, el
% algoritmo codicioso actualmente en uso en realidad minimiza la cantidad de
% camiones que se necesitan. Su prueba debe seguir el tipo de análisis que usamos
% para el problema de programación de intervalos: debe establecer la optimización
% de este algoritmo de empaquetamiento codicioso al identificar una medida bajo la
% cual "se mantiene por delante" de todas las demás soluciones.

\section*{Problema 4.8}
% Prueba por contradicción

%Suponga que le dan un grafo conectado G, donde los costos de las aristas son
%todos distintos. Demuestre que G tiene un árbol de expansión mínimo único.

\section*{Problema 4.9}

%Una de las motivaciones básicas detrás del problema del árbol de expansión
%mínimo es el objetivo de diseñar una red de expansión para un conjunto de nodos
%con un costo total mínimo. Aquí exploramos otro tipo de objetivo: diseñar una
%red de expansión en la que el borde más caro sea lo más barato posible.
%
%Específicamente, sea G = (V , E) un grafo conexo con n vértices, m aristas y
%costos de aristas positivos que puede suponer que son todos distintos. Sea T =
%(V , E ) un árbol generador de G; definimos el borde del cuello de botella de T
%como el borde de T con el mayor costo.
%
%Un árbol de expansión T de G es un árbol de expansión de cuello de botella
%mínimo si no hay un árbol de expansión T de G con un borde de cuello de botella
%más económico.
%
%(a) ¿Todo árbol cuello de botella mínimo de G es un árbol generador mínimo de G?
%Demostrar o dar un contraejemplo.
%
%(b) ¿Todo árbol generador mínimo de G es un árbol cuello de botella mínimo de G?
%Demostrar o dar un contraejemplo.


\section*{Problema 4.19}

%Un grupo de diseñadores de redes de la empresa de comunicaciones CluNet se
%enfrenta al siguiente problema. Tienen un grafo conexo G = (V , E), en el que
%los nodos representan sitios que quieren comunicarse. Cada borde e es un enlace
%de comunicación, con un ancho de banda disponible dado b e .
%
%Para cada par de nodos u, v ∈ V, quieren seleccionar un solo camino u-v P en el
%que este par se comunicará. La tasa de cuello de botella b(P) de esta ruta P es
%el ancho de banda mínimo de cualquier borde que contenga; es decir, b(P) = min
%e∈P b e .
%
%La mejor tasa de cuello de botella alcanzable para el par u, v en G es
%simplemente el máximo, sobre todos los caminos u-v P en G, del valor b(P).
%
%Se está volviendo muy complicado hacer un seguimiento de una ruta para cada par
%de nodos, por lo que uno de los diseñadores de redes hace una sugerencia audaz:
%
%Tal vez uno pueda encontrar un árbol de expansión T de G de modo que para cada
%par de nodos u, v, la única ruta u-v en el árbol realmente alcance la mejor tasa
%de cuello de botella alcanzable para u, v en G. (En otras palabras, incluso si
%podría elegir cualquier ruta u-v en todo el gráfico, no podría hacerlo mejor que
%la ruta u-v en T).
%
%Esta idea es interrumpida rotundamente en las oficinas de CluNet durante unos
%días, y hay una razón natural para el escepticismo: cada par de nodos puede
%querer una ruta de aspecto muy diferente para maximizar su tasa de cuello de
%botella; ¿Por qué debería haber un solo árbol que al mismo tiempo haga felices a
%todos? Pero después de algunos intentos fallidos de descartar la idea, la gente
%comienza a sospechar que podría ser posible.

\end{document}
