\documentclass{article}
\usepackage[spanish,es-tabla,es-nodecimaldot]{babel}
\usepackage{amsmath}
\usepackage[shortlabels]{enumitem}

\begin{document}
\title{Análisis y diseño de algoritmos. \\ Tarea 04}
\author{Ana Maritza Bello Yañez}
\maketitle
\setlength{\parindent}{0pt}
\setlength{\parskip}{1em}

\section*{Problema 4.3}
% Prueba por inducción
% Usted está asesorando a una empresa de camiones que hace una gran cantidad de
% envíos comerciales de paquetes entre Nueva York y Boston. El volumen es tan alto
% que tienen que enviar una cantidad de camiones cada día entre las dos
% ubicaciones. Los camiones tienen un límite fijo W sobre la cantidad máxima de
% peso que pueden transportar. Las cajas llegan a la estación de Nueva York una a
% una, y cada paquete i tiene un peso w_i. La estación de camiones es bastante
% pequeña, por lo que como máximo puede haber un camión en la estación en
% cualquier momento. La política de la empresa requiere que las cajas se envíen en
% el orden en que llegan; de lo contrario, un cliente podría molestarse al ver que
% una caja que llegó después de la suya llegó a Boston más rápido. Por el momento,
% la compañía está utilizando un algoritmo codicioso simple para empacar: empacan
% las cajas en el orden en que llegan, y cuando la siguiente caja no cabe, envían
% el camión por su camino.
% 
% Pero se preguntan si podrían estar usando demasiados camiones y quieren su
% opinión sobre si la situación se puede mejorar. Así es como están pensando. Tal
% vez se podría disminuir la cantidad de camiones necesarios enviando a veces un
% camión que estaba menos lleno, y de esta manera permitir que los próximos
% camiones estén mejor embalados.
% 
% Demuestre que, para un conjunto dado de cajas con pesos específicos, el
% algoritmo codicioso actualmente en uso en realidad minimiza la cantidad de
% camiones que se necesitan. Su prueba debe seguir el tipo de análisis que usamos
% para el problema de programación de intervalos: debe establecer la optimización
% de este algoritmo de empaquetamiento codicioso al identificar una medida bajo la
% cual "se mantiene por delante" de todas las demás soluciones.

No existe otra manera más eficiente de realizar los envíos ya que debemos de
cumplir que los paquetes se envien conforme van llegando a la empresa. Si
modificamos en orden de tal forma que los camiones no se vayan vacíos, entonces
no se cumple la condición. Tampoco podemos llenar dos camiones al mismo tiempo
porque necesitamos que solo haya uno en la empresa y tampoco podemos exceder la carga.


\section*{Problema 4.8}
% Prueba por contradicción

%Suponga que le dan un grafo conectado G, donde los costos de las aristas son
%todos distintos. Demuestre que G tiene un árbol de expansión mínimo único.

Podemos inducir un árbol de expansión mínima sobre el grafo $G$ ya sea con el
algoritmo de Prim o el de Kruskal.

\textbf{Caso 1. Algoritmo de Prim}

Comenzamos tomando un nodo cualquiera, nos vamos a fijar en cada una de sus
aristas y vamos a tomar la de menor peso.

Posteriormente del nodo visitado tomaremos la siguiente arista con menor peso, y
así sucesivamente. Dado que todas las aristas son distintas, a partir del primer
nodo visitado vamos a comenzar a agregar en order ascendente las aristas
adyacentes a partir de esa primera. Si hubiera alguna más pequeña que aún no se
ha visitado hasta este punto, entonces el algoritmo va a regresar al nodo cuya
arista sea de menor tamaño a la última agregada. Así solo habrá un solo árbol de
expansión mínimo, ya que los pesos de las aristas son únicos.

\textbf{Caso 1. Algoritmo de Kruskal}

Con el algoritmo de Kruskal comenzamos infiriendo nuestro árbol de expansión
sobre $G$ con la arista de costo mínimo. De las aristas restantes vamos a seguir
agregando la de menor tamaño. Dado que las aristas son de tamaños diferentes no
habrá dos que se repitan por lo que el algoritmo tomará el único camino de forma
ascendente y por lo tanto existirá un árbol de expansión mínima único.


\section*{Problema 4.9}

% Una de las motivaciones básicas detrás del problema del árbol de expansión
% mínimo es el objetivo de diseñar una red de expansión para un conjunto de nodos
% con un costo total mínimo. Aquí exploramos otro tipo de objetivo: diseñar una
% red de expansión en la que el borde más caro sea lo más barato posible.
% 
% Específicamente, sea G = (V , E) un grafo conexo con n vértices, m aristas y
% costos de aristas positivos que puede suponer que son todos distintos. Sea T =
% (V , E ) un árbol generador de G; definimos el borde del cuello de botella de T
% como el borde de T con el mayor costo.
% 
% Un árbol de expansión T de G es un árbol de expansión de cuello de botella
% mínimo si no hay un árbol de expansión T de G con un borde de cuello de botella
% más económico.
MBN: Minimum Bottleneck Tree.

MST: Minimum Spanning Tree.

(a) ¿Todo MBN(G) es un MST(G)?
% Demostrar o dar un contraejemplo.

Si, ya que si $T(G)$ es el $MBN(G)$, quiere decir que no existe otro árbol con
cuello de botella más pequeño, y para el MST también se cumple esta condición.

% 
(b) ¿Todo MST(G) es un MBT(G)?
% Demostrar o dar un contraejemplo.

Como en el caso anterior, el $MBT(G) = MST(G)$

\section*{Problema 4.19}

%Un grupo de diseñadores de redes de la empresa de comunicaciones CluNet se
%enfrenta al siguiente problema. Tienen un grafo conexo G = (V , E), en el que
%los nodos representan sitios que quieren comunicarse. Cada borde e es un enlace
%de comunicación, con un ancho de banda disponible dado b e .
%
%Para cada par de nodos u, v ∈ V, quieren seleccionar un solo camino u-v P en el
%que este par se comunicará. La tasa de cuello de botella b(P) de esta ruta P es
%el ancho de banda mínimo de cualquier borde que contenga; es decir, b(P) = min
%e∈P b e .
%
%La mejor tasa de cuello de botella alcanzable para el par u, v en G es
%simplemente el máximo, sobre todos los caminos u-v P en G, del valor b(P).
%
%Se está volviendo muy complicado hacer un seguimiento de una ruta para cada par
%de nodos, por lo que uno de los diseñadores de redes hace una sugerencia audaz:
%
%Tal vez uno pueda encontrar un árbol de expansión T de G de modo que para cada
%par de nodos u, v, la única ruta u-v en el árbol realmente alcance la mejor tasa
%de cuello de botella alcanzable para u, v en G. (En otras palabras, incluso si
%podría elegir cualquier ruta u-v en todo el gráfico, no podría hacerlo mejor que
%la ruta u-v en T).
%
%Esta idea es interrumpida rotundamente en las oficinas de CluNet durante unos
%días, y hay una razón natural para el escepticismo: cada par de nodos puede
%querer una ruta de aspecto muy diferente para maximizar su tasa de cuello de
%botella; ¿Por qué debería haber un solo árbol que al mismo tiempo haga felices a
%todos? Pero después de algunos intentos fallidos de descartar la idea, la gente
%comienza a sospechar que podría ser posible.
En este caso, necesitamos encontrar el camino con mayor ancho posible,
en otras palabras, queremos encontrar el árbol de expansión para el cual, la
suma de sus pesos no sea el mínimo posible, sino el máximo.

Para encontrarlo podemos usar el algoritmo del árbol de expansión mínima, en
este caso el de Prim, pero en vez de aceptar los pesos mínimos, acepta los pesos
máximos.

Este algoritmo acepta un grafo de pesos o costos con $n$ nodos. En el algoritmo,
$T$ es el conjunto de aristas que conforman el árbol y $P$ es el conjunto de
nodos visitados o procesados con las aristas de $T$, $N$ el conjunto de nodos
del grafo a ser procesados. \\
\\
\textbf{PASO 1}

Selecciona un nodo inicial $U$.

Establece $ P={U} $ y $T=\emptyset$\\
\\
\textbf{PASO 2}

\textbf{\textit{while }} $N$ no esté vacía \{
    \begin{itemize}
        \item Escoge la arista mas pesada o de mayor costo
        \item Agrega la arista a $T$
        \item Agrega el nodo a $P$ y quítalo de $N$.
    \end{itemize}
    \}
\textbf{\textit{end while }}\\
\\
\textbf{PASO 3}

\textbf{\textit{if }} $|P| < n$
El grafo no está conectado y por lo tanto no tiene un árbol de expansión máximo.
Regresar al paso 2.
    
\textbf{\textit{elif }}
Las aristas en $T$ y los nodos visitados forman un árbol de expansión máxima.

\textbf{\textit{endif }}

\end{document}
