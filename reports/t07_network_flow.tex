\documentclass{article}
\usepackage[spanish,es-tabla,es-nodecimaldot]{babel}
\usepackage{amsmath}
\usepackage[shortlabels]{enumitem}

\begin{document}
\title{Análisis y diseño de algoritmos. \\ Tarea 07}
\author{Ana Maritza Bello Yañez}
\maketitle
\setlength{\parindent}{0pt}
\setlength{\parskip}{1em}

\section*{Problema 7.01}

\begin{enumerate}

\item Enumere todos los cortes mínimos de $s-t$ en la red de flujo representada en
la figura 7.24. La capacidad de cada borde aparece como una etiqueta al lado del
borde.

\textbf{Solución}


$\text{cut}_1 = \{s-u, s-v\}$;     $c_1(X,Y) = 2$


$\text{cut}_2 = \{u-t,v-t\}$;      $c_2(X,Y) = 2$


$\text{cut}_3 = \{ u-t, u-v, s-v \}$;  $c_3(X,Y) = 3$


$\text{cut}_4 = \{ s-u, v-t\}$;    $c_4(X,Y) = 2$

Por lo tanto, el corte mínimo es 2.

\item ¿Cuál es la capacidad mínima de un corte $s-t$ en la red de flujo de la
figura 7.25? Nuevamente, la capacidad de cada borde aparece como una etiqueta al
lado del borde.

\textbf{Solución}


$\text{cut}_1 = \{s-u, s-v\} = \{2,4\}$;     $c_1(X,Y) = 6$


$\text{cut}_2 = \{u-t,v-t\} = \{4,6\}$;      $c_2(X,Y) = 10$


$\text{cut}_3 = \{ u-t, u-v, s-v \} = \{4,6,4\}$;  $c_3(X,Y) = 14$


$\text{cut}_4 = \{ s-u, v-t\} = \{2,2\}$;    $c_4(X,Y) = 4$

Por lo tanto, $min\_cut(X,Y) = \{4\}$

\end{enumerate}    

\section*{Problema 7.07}

Considere un conjunto de clientes de computación móvil en una determinada
ciudad, cada uno de los cuales necesita estar conectado a una de varias
\textit{estaciones base posibles}. Supondremos que hay n clientes, con la
posición de cada cliente especificada por sus coordenadas $(x, y)$ en el plano.
También hay $k$ estaciones base; la posición de cada uno de ellos también se
especifica mediante coordenadas $(x, y)$.

Para cada cliente, deseamos conectarlo exactamente a una de las estaciones base.

Nuestra elección de conexiones está restringida de la siguiente manera.

\textbf{Solución}

Hay un \textit{parámetro de rango} $r$: un cliente solo se puede conectar a una
estación base que se encuentra dentro de la distancia $r$. También hay un
\textit{parámetro de carga} $L$: no se pueden conectar más de $L$ clientes a una
sola estación base.

Su objetivo es diseñar un algoritmo de tiempo polinomial para el siguiente
problema. Dadas las posiciones de un conjunto de clientes y un conjunto de
estaciones base, así como los parámetros de alcance y carga, decida si todos los
clientes pueden conectarse simultáneamente a una estación base, sujeto a las
condiciones de alcance y carga en el párrafo anterior.

\section*{Problema 7.10}

Suponga que le dan un grafo dirigido $G = (V , E)$, con una capacidad entera
positiva $c_e$ en cada arista $e$, una fuente $s \in V$ y un sumidero $t \in V$.
También le dan un flujo máximo $s-t$ en $G$, definido por un valor de flujo
$f_e$ en cada arista $e$. El flujo $f$ es acíclico: no hay ciclo en $G$ en el
que todos los bordes lleven un flujo positivo. El flujo $f$ también tiene un
valor entero.

Ahora supongamos que elegimos una arista específica $e \cdot \in E$ y reducimos su
capacidad en 1 unidad. Muestre cómo encontrar un flujo máximo en el gráfico
capacitado resultante en el tiempo $O(m + n)$, donde $m$ es el número de aristas en
$G$ y $n$ es el número de nodos.

\textbf{Solución}

\section*{Problema 7.12}

Considere el siguiente problema. Tiene una red de flujo con aristas de capacidad
unitaria: consiste en un gráfico dirigido $G = (V , E)$, una fuente $s \in V$ y
un sumidero $t \in V$; y $c_e = 1$ para cada $e \in E$. También se te da un
parámetro $k$.

El objetivo es eliminar los bordes $k$ para reducir el flujo máximo de $s-t$ en
$G$ tanto como sea posible. En otras palabras, debes encontrar un conjunto de
aristas $F \subseteq E$ tal que $|F| = k$ y el flujo máximo de $s-t$ en $G = (V
, E - F)$ es lo más pequeño posible sujeto a esto.

Proporcione un algoritmo de tiempo polinomial para resolver este problema.

% \textbf{Solución}

% \section*{Problema 7.16}
% En los eufóricos primeros días de la Web, a la gente le gustaba afirmar que gran
% parte del enorme potencial de una empresa como Yahoo! estaba en los “globos
% oculares”, el simple hecho de que millones de personas miran sus páginas todos
% los días. Además, al convencer a las personas de que registren datos personales
% en el sitio, un sitio como Yahoo! puede mostrar a cada usuario un anuncio
% extremadamente específico cada vez que visita el sitio, de una manera que las
% cadenas de televisión o las revistas no podrían igualar. Entonces, si un usuario
% le ha dicho a Yahoo! que él o ella es un estudiante de informática de la
% Universidad de Cornell de 20 años de edad, el sitio puede presentar un anuncio
% publicitario para apartamentos en Ithaca, Nueva York; por otro lado, si él o
% ella es un banquero de inversiones de 50 años de edad de Greenwich, Connecticut,
% el sitio puede mostrar un anuncio publicitario que presente Lincoln Town Cars en
% su lugar.
% 
% Pero decidir qué anuncios mostrar a qué personas implica un cálculo serio detrás
% de escena. Suponga que los administradores de un sitio web popular han
% identificado $k$ grupos demográficos distintos $G_1, G_2, . . . , G_k$. (Estos
% grupos pueden superponerse; por ejemplo, $G_1$ puede ser igual a todos los
% residentes del estado de Nueva York y $G_2$ puede ser igual a todas las personas
% con un título en informática). El sitio tiene contratos con $m$ anunciantes
% diferentes, para mostrar un cierto número de copias de sus anuncios a los
% usuarios del sitio. Así es como se ve el contrato con el $i$-ésimo anunciante.
% 
% \begin{itemize}
% 
% \item  Para un subconjunto $X_i \subseteq \{G 1 , . . . , G k \}$ de los grupos
% demográficos, el anunciante $i$ desea que sus anuncios se muestren solo a los
% usuarios que pertenecen al menos a uno de los grupos demográficos del conjunto
% $X_i$.
% 
% \item Para un número r i , el anunciante i quiere que sus anuncios se muestren
% al menos a r i usuarios cada minuto.
% 
% \end{itemize}
% 
% Ahora considere el problema de diseñar una buena política publicitaria, una
% forma de mostrar un solo anuncio a cada usuario del sitio. Supongamos que en un
% minuto dado hay n usuarios visitando el sitio. Como tenemos información de
% registro de cada uno de estos usuarios, sabemos que el usuario j (para $j = 1, 2,
% . . . , n)$ pertenece a un subconjunto $U_j \subseteq \{G_1 , . . . , G_k \}$ de los grupos
% demográficos. El problema es: ¿hay alguna manera de mostrar un solo anuncio a
% cada usuario para que los contratos del sitio con cada uno de los m anunciantes
% se cumplan por ese minuto? (Es decir, para cada $i = 1, 2, . . . , m$, ¿puede al
% menos $r_i$ de los $n$ usuarios, cada uno perteneciente a al menos un grupo
% demográfico en $X_i$ , recibir un anuncio proporcionado por el anunciante $i$?)
% 
% Proporcione un algoritmo eficiente para decidir si esto es posible y, de ser
% así, elegir un anuncio para mostrar a cada usuario.

\end{document}