\documentclass{article}
\usepackage[spanish,es-tabla,es-nodecimaldot]{babel}
\usepackage{amsmath}
\usepackage[shortlabels]{enumitem}
\usepackage{tikz}
\usepackage{graphicx}

\setlength{\parindent}{0pt}
\setlength{\parskip}{1em}


\begin{document}
\title{Análisis y diseño de algoritmos. \\ Tarea 03}
\author{Ana Maritza Bello Yañez}
\maketitle

\section*{Problema 3.5}

% Un árbol binario es un árbol con raíz en el que cada nodo tiene como máximo dos
% hijos. Muestre por inducción que en cualquier árbol binario el número de nodos
% con dos hijos es exactamente uno menos que el número de hojas.

Sean $n_{total}(T)$ el número de hojas de un árbol binario $T$, y sea
$n_{ancestros}(T)$ el número de nodos con dos hijos del árbol binario $T$.

Debemos demostrar que $n_{ancestros}(T)$ = $n_{total}(T) - 1$.

\textbf{Caso Base:} $n_{total} = 1$, por lo tanto $n_{ancestros}(T)= 1 - 1$, lo
que quiere decir que el árbol tiene una sola hoja y no existen nodos con dos
hijos.

Ahora, sea un árbol binario con más de un nodo, entonces existe un nodo
\textit{hoja}, $v$, que no es la raíz, ya que $T$ tiene más de un nodo.
Por lo tanto existe un nodo $u$ que es \textit{ancestro} de $v$.

\begin{figure}[h!]
    \begin{center}		
    \begin{tikzpicture}[level distance=1cm, level 1/.style={sibling distance=3cm},
      level 2/.style={sibling distance=2cm}, every node/.style={circle, draw,
      align=center} ]
          \centering
          \node[circle,draw]{}
          child{
            node{}
          }
        child{
          node{$u$} child{
          node{$w$}} child{node{$v$}
          }
        };
        \end{tikzpicture}
    \end{center}
    \caption{Representación del árbol binario $T$ con los nodos $u$, $v$ y $w$.}
\end{figure}

Si eliminamos el nodo $v$ del árbol binario $T$, tenemos dos casos:

\textbf{Caso 1.} Que en el árbol resultante $T'$, el nodo ancestro $u$, no tenga
otro nodo hijo, por lo tanto el nodo $u$ se convierte en un nodo \textit{hoja}. 

\begin{figure}[h!]
    \begin{center}		
    \begin{tikzpicture}[level distance=1cm, level 1/.style={sibling distance=3cm},
      level 2/.style={sibling distance=2cm}, every node/.style={circle, draw,
      align=center} ]
          \centering
          \node[circle,draw]{}
          child{
            node{1}
          }
        child{
          node{$u$}
        };
    \end{tikzpicture}
    \end{center}
    \caption{Representación del árbol binario recortado $T'$, donde el nodo $u$ no
    tiene nodos hijos, por lo tanto se convierte en un nodo hoja.}
\end{figure}

En este caso $n_{total}(T') = m_{ancestros}(T') + 1$ o dicho de otra forma
$n_{ancestros}(T') = n_{total}(T') - 1$

\textbf{Caso 2.} Que en el árbol resultante $T''$, el nodo ancestro $u$, tenga
otro nodo hijo $w$, en cuyo caso: $n_{ancestros}(T'') = n_{total}(T'') - 1$, ó dicho de
otra forma; $n_{total}(T'') = m_{ancestros}(T'') + 1$

\begin{figure}[h!]
    \begin{center}		
    \begin{tikzpicture}[level distance=1cm, level 1/.style={sibling distance=3cm},
      level 2/.style={sibling distance=2cm}, every node/.style={circle, draw,
      align=center} ]
          \centering
          \node[circle,draw]{}
          child{
            node{}
          }
        child{
          node{$u$} child{
          node{$w$}}
        };
    \end{tikzpicture}
    \end{center}
    \caption{Representación del árbol binario recortado $T''$, donde el nodo $u$ tiene
    un nodo hijo $w$.}
\end{figure}

Por lo que para los tres casos se cumple que:

$n_{total}(T) = n_{total}(T') =
n_{total}(T'')$

o bien; 

$n_{ancestros}(T) = n_{ancestros}(T') = n_{ancestros}(T'')$

\section*{Problema 3.6}

% Tenemos un grafo conexo G = (V , E) y un vértice específico u ∈ V. Supongamos
% que calculamos un árbol de búsqueda primero en profundidad con raíz en u, y
% obtenemos un árbol T que incluye todos los nodos de G. Supongamos que luego
% calculamos un árbol de búsqueda primero en anchura con raíz en u, y obtenga el
% mismo árbol T. Demuestre que G = T. (En otras palabras, si T es tanto un árbol
% de búsqueda primero en profundidad como un árbol de búsqueda primero en anchura
% con raíz en u, entonces G no puede contener ninguna arista que no pertenezca a
% T.)

Supongamos que nuestro grafo $G$ tiene una arista en $e = {x,y} \not\in T$.

Ya que $T$ es un árbol inducido por DFS, alguno de los dos extremos debe ser
ancestro del otro.

* Digamos que $x$ es un ancestro de $y$.

Ya que $T$ es un árbol inducido por BFS, la distancia entre los dos nodos desde
$u$ en $T$, pueden diferir a lo máximo por uno.

Pero si $x$ es ancestro de $y$, y la distancia de $u$ a $x$ en $T$ es a lo más
uno mayor que la distancia desde $u$ hasta $x$, entonces $x$ debe de ser
ancestro directo de $b$ en $T$.

Por lo que si $\{x,y\} \in T$ contradice nuestra asunsión inicial que de
$\{x,y\} \not\in T$

\section*{Problema 3.7}

% Unos amigos tuyos trabajan en redes inalámbricas y actualmente están estudiando
% las propiedades de una red de n dispositivos móviles. A medida que los
% dispositivos se mueven (en realidad, a medida que sus dueños se mueven),
% definen un gráfico en cualquier momento de la siguiente manera: hay un nodo que
% representa cada uno de los n dispositivos, y hay un borde entre el dispositivo i
% y el dispositivo j si las ubicaciones físicas de i y j no están separadas más de
% 500 metros. (Si es así, decimos que i y j están "en el rango" uno del otro).
% 
% Les gustaría que la red de dispositivos estuviera conectada en todo momento, por
% lo que restringieron el movimiento de los dispositivos para satisfacer la
% siguiente propiedad: en todo momento, cada dispositivo i está dentro de los 500
% metros de al menos menos n/2 de los otros dispositivos. (Supondremos que n es un
% número par).
% 
% Lo que les gustaría saber es: ¿Esta propiedad por sí sola garantiza que la red
% permanecerá conectada?
% 
% Aquí hay una forma concreta de formular la pregunta como una afirmación sobre
% grafos:
% 
% Afirmación: Sea G un grafo de n nodos, donde n es un número par. Si cada nodo
% de G tiene grado al menos n/2, entonces G es conexo.
% 
% Decide si crees que la afirmación es verdadera o falsa y proporciona una prueba
% de la afirmación o de su negación.

Sea un grafo $G$ donde $\{\forall u \in V | deg(u) \geq (n/2) \}$, debemos demostrar
que $G$ está conectado.

\textbf{Demostración por contradicción}

Supongamos que $G$ no está conectado, entonces tenemos dos casos:

\textbf{Caso 1. Cuando son del mismo tamaño.}
Sea $V$ el conjunto de nodos cuya cantidad de nodos conectados es la más
pequeña, entonces $| V | \leq (n/2)$, ya que al menos dos nodos están conectados.

\textbf{Caso 2. Cuando uno es más grande que el otro.}
Si $|V|$ es mayor, entonces se debe de cumplir que $|V|-1 \leq (n/2)-1 \leq
(n/2)$, lo cual es una contradicción.

Por lo tanto, el grafo si está conectado.


\section*{Problema 3.9}

% Hay una intuición natural de que dos nodos que están muy separados en una red de
% comunicación, separados por muchos saltos, tienen una conexión más tenue que dos
% nodos que están muy juntos. Hay una serie de resultados algorítmicos que se
% basan en cierta medida en diferentes formas de precisar esta noción. Aquí hay
% uno que involucra la susceptibilidad de las rutas a la eliminación de nodos.
% 
% Suponga que un gráfico no dirigido de n nodos G = (V , E) contiene dos nodos s y
% t tales que la distancia entre s y t es estrictamente mayor que n/2. Muestre que
% debe existir algún nodo v, que no sea igual a s o t, tal que al eliminar v de G
% se destruyen todos los caminos s-t. (En otras palabras, el gráfico obtenido de G
% eliminando v no contiene un camino de s a t).
% 
% Proporcione un algoritmo con tiempo de ejecución O(m + n) para encontrar dicho
% nodo v.

\end{document}
