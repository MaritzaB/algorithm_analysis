\documentclass{article}
\usepackage[spanish,es-tabla,es-nodecimaldot]{babel}
\usepackage{amsmath}
\usepackage[shortlabels]{enumitem}

\begin{document}
\title{Análisis y diseño de algoritmos. \\ Tarea 06}
\author{Ana Maritza Bello Yañez}
\maketitle
\setlength{\parindent}{0pt}
\setlength{\parskip}{1em}

\section*{Problema 6.1}
% Sea G = (V , E) un grafo no dirigido con n nodos. Recuerde que un subconjunto de
% los nodos se llama conjunto independiente si dos de ellos no están unidos por
% una arista. En general, es difícil encontrar grandes conjuntos independientes;
% pero aquí veremos que se puede hacer de manera eficiente si el gráfico es lo
% suficientemente "simple".
% 
% Llame a un gráfico G = (V , E) un camino si sus nodos se pueden escribir como v
% 1 , v 2 , . . . , v n , con una arista entre vi y v j si y solo si los números i
% y j difieren exactamente en 1. Con cada nodo v i , asociamos un peso entero
% positivo w i .
% 
% Considere, por ejemplo, la ruta de cinco nodos dibujada en la figura 6.28. Los
% pesos son los números dibujados dentro de los nodos.
% 
% El objetivo de esta pregunta es resolver el siguiente problema:
% 
% Encuentre un conjunto independiente en un camino G cuyo peso total sea lo más
% grande posible.

\begin{enumerate}
\item Dé un ejemplo para mostrar que el siguiente algoritmo no siempre encuentra
un conjunto independiente de peso total máximo.

\textbf{Solucion:} Supongamos un grafo que tiene 3 pesos diferentes, siendo el
de el medio el de mayor peso. Por ejemplo: 5 - 6 - 4, con el algoritmos greedy
tomaríamos el nodo 6=6, sin embargo el \textit{independent set} consiste del 5 y
el 4.

\item Dé un ejemplo para mostrar que el siguiente algoritmo tampoco encuentra
siempre un conjunto independiente de peso total máximo.

\textbf{Solución: }

Supongamos un grafo con con los nodos: 2-7-6-4. En este caso tenemos dos pares
de conjuntos independientes posibles, el 2 y 6 y el otro conjunto 7 y 4. El
algoritmo podría tomar el primer conjunto, sin embargo el de mayor peso es el de
los nodos 7 y 4.
\end{enumerate}

\section*{Problema 6.2}

\section*{Problema 6.8}

\end{document}
