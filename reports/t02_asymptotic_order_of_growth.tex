\documentclass{article}
\usepackage[spanish,es-tabla,es-nodecimaldot]{babel}
\usepackage{amsmath}
\usepackage[shortlabels]{enumitem}
\setlength{\parindent}{0pt}
\setlength{\parskip}{1em}


\begin{document}
\title{Análisis y diseño de algoritmos. \\ Tarea 02}
\author{Ana Maritza Bello Yañez}
\maketitle

\section*{Problema 2.2}

Si una computadora realiza $ 10^{10} $ operaciones por segundo, entonces el
total de operaciones por hora = $3.6 * 10^{13} $.

\begin{enumerate}[(a)]
\item Si el total de operaciones es igual a $3.6 * 10^{13} $, entonces con un
algoritmo de tiempo $n^2$ podríamos realizar:

    \begin{equation}
        n = \sqrt{3.6 * 10^{13}} = 6000000 \mathrm{\ operaciones\ por\ hora}
    \end{equation}

\item Para un algoritmo $T(n^3)$ podemos realizar:
    \begin{equation}
        n = \sqrt[3]{3.6 * 10^{13}} = 33019.2725 \mathrm{\ operaciones\ por\ hora}
    \end{equation}

\item Para un algoritmo de $T(100 n^2)$ podemos realizar:
    \begin{equation}
        n = \sqrt[2]{\frac{(3.6 * 10^{13})}{100}} = 600000 
    \end{equation}

\item Para $T(n \log{n})$ podemos realizar:

\item Para $T(2^n)$ podemos realizar:
    \begin{equation}
        n = \frac{\log(3.6 * 10^{13})}{\log(2)} = 45.033 \mathrm{\ operaciones\ por\ hora}
    \end{equation}

\item Para $T(2^{2^n})$ podemos realizar:
    \begin{equation}
        n = \frac{\log(\frac{\log(3.6 * 10^{13}))}{\log(2)}}{\log(2)} = 5.492 \mathrm{\ operaciones\ por\ hora}
    \end{equation}

\end{enumerate}

\section*{Problema 2.6}

\begin{enumerate}[(a)]
    
\item El algoritmo consta de los ciclos anidados. El primero de ellos corre
hasta $n$, por lo que ese ciclo vale $O(n)$. El segundo ciclo que está anidado
en el primero, corre hasta las $n$ veces del de afuera y genera una matriz
bi-dimensional sumando un número al iterador de afuera, por lo que el segundo
ciclo corre a lo máximo a $n^2$. El tiempo total de este algoritmo es de
$O(n^3)$

\item 


\end{enumerate}


\end{document}
