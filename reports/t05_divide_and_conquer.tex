\documentclass{article}
\usepackage[spanish,es-tabla,es-nodecimaldot]{babel}
\usepackage{amsmath}
\usepackage[shortlabels]{enumitem}

\begin{document}
\title{Análisis y diseño de algoritmos. \\ Tarea 05}
\author{Ana Maritza Bello Yañez}
\maketitle
\setlength{\parindent}{0pt}
\setlength{\parskip}{1em}

\section*{Problema 5.1}

% Está interesado en analizar algunos datos difíciles de obtener de dos bases de
% datos separadas. Cada base de datos contiene n valores numéricos, por lo que hay
% 2n valores en total, y puede suponer que no hay dos valores iguales.
% Le gustaría determinar la mediana de este conjunto de 2n valores, que
% definiremos aquí como el enésimo valor más pequeño.
% Sin embargo, la única forma de acceder a estos valores es mediante consultas a
% las bases de datos. En una sola consulta, puede especificar un valor k para una
% de las dos bases de datos, y la base de datos elegida devolverá el k-ésimo valor
% más pequeño que contiene. Dado que las consultas son costosas, le gustaría
% calcular la mediana utilizando la menor cantidad de consultas posible.
% Proporcione un algoritmo que encuentre el valor de la mediana utilizando como
% máximo consultas O (log n).

Si tenemos dos bases de datos de $n$ elementos, y el $k$-ésimo elemento es la
media de cada una de las bases de datos, entonces tenemos algo como lo
siguiente:

\begin{center}
    $DB_1:$
\end{center}

\begin{center}
    $DB_2:$
\end{center}

\section*{Problema 5.3}

% Suponga que está consultando para un banco que está preocupado por la detección
% de fraudes, y acuden a usted con el siguiente problema. Tienen una colección de
% n tarjetas bancarias que han confiscado, sospechando que se están utilizando en
% fraudes. Cada tarjeta bancaria es un pequeño objeto de plástico que contiene una
% banda magnética con algunos datos encriptados y corresponde a una cuenta única
% en el banco. Cada cuenta puede tener muchas tarjetas bancarias correspondientes,
% y diremos que dos tarjetas bancarias son equivalentes si corresponden a la misma
% cuenta.
% Es muy difícil leer directamente el número de cuenta de una tarjeta bancaria,
% pero el banco tiene un "probador de equivalencias" de alta tecnología que toma
% dos tarjetas bancarias y, después de realizar algunos cálculos, determina si son
% equivalentes.
% Su pregunta es la siguiente: entre la colección de n cartas, ¿hay un conjunto de
% más de n/2 de ellas que sean todas equivalentes entre sí?
% Suponga que las únicas operaciones factibles que puede hacer con las tarjetas
% son elegir dos de ellas y conectarlas al probador de equivalencia. Muestre cómo
% decidir la respuesta a su pregunta con solo invocaciones O (n log n) del
% probador de equivalencia.

\section*{Problema 5.6}

% Considere un árbol binario completo T de n nodos, donde n = 2^d-1 para alguna
% d.
% Cada nodo v de T está etiquetado con un número real x_v . Puede suponer que los
% números reales que etiquetan los nodos son todos distintos. Un nodo v de T es un
% mínimo local si la etiqueta x_v es menor que la etiqueta x_w para todos los
% nodos w que están unidos a v por una arista.
% Se le proporciona un árbol binario T tan completo, pero el etiquetado solo se
% especifica de la siguiente manera implícita: para cada nodo v, puede determinar
% el valor x_v sondeando el nodo v. Muestre cómo encontrar un mínimo local de T
% usando solo O(log n) sondea a los nodos de T.

\end{document}
