\documentclass{article}
\usepackage[spanish,es-tabla,es-nodecimaldot]{babel}
\usepackage{amsmath}
\usepackage[shortlabels]{enumitem}

\begin{document}
\title{Análisis y diseño de algoritmos. \\ Tarea 05}
\author{Ana Maritza Bello Yañez}
\maketitle
\setlength{\parindent}{0pt}
\setlength{\parskip}{1em}

\section*{Problema 5.1}

% Está interesado en analizar algunos datos difíciles de obtener de dos bases de
% datos separadas. Cada base de datos contiene n valores numéricos, por lo que hay
% 2n valores en total, y puede suponer que no hay dos valores iguales.
% Le gustaría determinar la mediana de este conjunto de 2n valores, que
% definiremos aquí como el enésimo valor más pequeño.
% Sin embargo, la única forma de acceder a estos valores es mediante consultas a
% las bases de datos. En una sola consulta, puede especificar un valor k para una
% de las dos bases de datos, y la base de datos elegida devolverá el k-ésimo valor
% más pequeño que contiene. Dado que las consultas son costosas, le gustaría
% calcular la mediana utilizando la menor cantidad de consultas posible.
% Proporcione un algoritmo que encuentre el valor de la mediana utilizando como
% máximo consultas O (log n).

Si tenemos dos bases de datos de $n$ elementos, y el $k$-ésimo elemento es la
media de cada una de las bases de datos, entonces tenemos algo como lo
siguiente:

\begin{center}
    $DB_1:$
    \begin{tabular}{c|c|c|c|c|c|c}
        $a$ & $b$ & ... & $k$ & ... & $n-1$ & $n$
    \end{tabular}
\end{center}

\begin{center}
    $DB_2:$
    \begin{tabular}{c|c|c|c|c|c|c}
        $a$ & $b$ & ... & $k$ & ... & $n-1$ & $n$
    \end{tabular}
\end{center}

Es decir, $k = n/2$ para cada una de las bases de datos. Por lo que $DB_1(k)$ es
la media de la primera base de datos y $BD_2(k)$ la media para la segunda base
de datos. Si comparamos las medias de cada una de las bases de datos,
forzosamente una de las dos medias será mayor que la otra, ya que una de las
condiciones es que los números no se repiten. Por lo tanto tendremos los
siguientes casos:

Caso 1: $DB_1(k) > DB_2(k)$

Caso 2: $DB_2(k) > DB_1(k)$

Si tenemos el caso 1, entonces quiere decir que la mediana de $DB_2$ es mayor
que todos los números de la primera mitad de $DB_2$, incluyendo la mediana.
También, es mayor que $DB_2(k-1)$, por lo que si combinaramos las bases de
datos, $DB_2(k)$ ocuparía el lugar $2k$ de la lista combinada.

Entonces, podemos eliminar la primera mitad de elementos de $DB_1$ y la mitad de
elementos mayores a $DB_2$. Con esto nos queda una lista con un número de
elementos $n$ igual que la que teníamos originalmente. Por lo que podremos usar
recursión sobre esa misma lista.

\pagebreak

\begin{verbatim}
    median(a,b,c)
        if n=1 then return min(A(a+k),B(b+k)) //caso base
        k=[1/2 n]
        if A(a+k)<B(b+k)
            then return median(k, a+[1/2n], b)
            else return median(k, a, b+[1/2n])
\end{verbatim}

\section*{Problema 5.3}

% Suponga que está consultando para un banco que está preocupado por la detección
% de fraudes, y acuden a usted con el siguiente problema. Tienen una colección de
% n tarjetas bancarias que han confiscado, sospechando que se están utilizando en
% fraudes. Cada tarjeta bancaria es un pequeño objeto de plástico que contiene una
% banda magnética con algunos datos encriptados y corresponde a una cuenta única
% en el banco. Cada cuenta puede tener muchas tarjetas bancarias correspondientes,
% y diremos que dos tarjetas bancarias son equivalentes si corresponden a la misma
% cuenta.
% Es muy difícil leer directamente el número de cuenta de una tarjeta bancaria,
% pero el banco tiene un "probador de equivalencias" de alta tecnología que toma
% dos tarjetas bancarias y, después de realizar algunos cálculos, determina si son
% equivalentes.
% Su pregunta es la siguiente: entre la colección de n cartas, ¿hay un conjunto de
% más de n/2 de ellas que sean todas equivalentes entre sí?
% Suponga que las únicas operaciones factibles que puede hacer con las tarjetas
% son elegir dos de ellas y conectarlas al probador de equivalencia. Muestre cómo
% decidir la respuesta a su pregunta con solo invocaciones O (n log n) del
% probador de equivalencia.

\section*{Problema 5.6}

% Considere un árbol binario completo T de n nodos, donde n = 2^d-1 para alguna
% d.
% Cada nodo v de T está etiquetado con un número real x_v . Puede suponer que los
% números reales que etiquetan los nodos son todos distintos. Un nodo v de T es un
% mínimo local si la etiqueta x_v es menor que la etiqueta x_w para todos los
% nodos w que están unidos a v por una arista.
% Se le proporciona un árbol binario T tan completo, pero el etiquetado solo se
% especifica de la siguiente manera implícita: para cada nodo v, puede determinar
% el valor x_v sondeando el nodo v. Muestre cómo encontrar un mínimo local de T
% usando solo O(log n) sondea a los nodos de T.

El caso base de este problema sería cuando tengamos un árbol de un nivel
solamente, es decir, la raíz con sus dos nodos hijos. En este caso si la raíz es
menor que sus dos hijos, entonces es un mínimo local. En caso de que cualquiera
de sus dos hijos sea menor, entonces seguirá con el nodo hijo más pequeño y así
hasta que termine en un nodo $v$ donde este sea menor que sus nodos hijos, o
bien hasta alcanzar $w$.

\end{document}
